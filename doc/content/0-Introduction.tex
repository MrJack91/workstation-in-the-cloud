
\chapter{Einleitung}

\section{Ziele}

In der vorliegenden Arbeit sollen Vor-/ und Nachteile von der Cloud...

Steht ein Umdenken von der bisherigen „on-premise“ Strategie, zur Cloud-Philosophie an?
\begin{itemize}
	\item Können, in Anbetracht von Kosten und Aufwand, verfügbare Ressourcen effizient, (aber gerecht) zur Verfügung gestellt werden?
	\item Welche Software für solche Ansprüche ist momentan auf dem Markt führend?
	\item Welche Hardware wird für die Bereitstellung der Cloud benötigt?
\end{itemize}


\section{Begründung}
Die meisten Menschen arbeiten gegenwärtig mit einer traditionellen „on-premise“ Computerlösung.
Ein Grossteil der Rechenleistung bleibt ungenutzt, da nur zu Peaks die volle Leistung benötigt wird.
Spart man Geld bei den Workstations, muss der Benutzer oft warten, wenn rechen-intensive Arbeiten durchgeführt werden. 
Mit einer Cloudlösung soll es möglich sein, dem User ständig so viel Rechenpower bereitzustellen, wie er benötigt.
Dies spart Ressourcen ein, da die Leistung dann zwischen den Benutzern geteilt werden kann.

In letzter Zeit haben die Möglichkeiten der Virtualisierung stark zugenommen.


\section{Abgrenzung}


