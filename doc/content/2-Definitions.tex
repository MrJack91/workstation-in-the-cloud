\chapter{Allgemein zur Cloud}

\section{Definitionen}
\label{sec:cloud:definition}
% allgemeine Definition von Cloud

Der Begriff \textit{\Gls{glos:cloudLabel}} resp. \textit{\Gls{glos:cloudLabel} Computing} wird sehr häufig gebraucht (eine Google Suche findet 136 Mio. Resultate zum zweit genannten Begriff).
Trotz dem vielen Wortgebrauch ist eine klare Definition kaum bekannt.

Folgend drei Beispiele, die mögliche Definitionen aufzeigen sollen.

\begin{quote}
	Die \textit{\Gls{glos:cloudLabel}} ist eines der ältesten Sinnbilder der Informationstechnik und steht als solches für Rechnernetze, deren Inneres unbedeutend oder unbekannt ist. \textit{- Quelle: The Wall Street Journal}  \footcite{The_Internet_Industry_Is_on_a_Cloud_--_Whatever_That_May_Mean_-_WSJ_2014-10-03}
\end{quote}

Oder ein wenig konkreter:

\begin{quote}
	Unter \textit{\Gls{glos:cloudLabel} Computing} (deutsch etwa: Rechnen in der Wolke) versteht man das Speichern von Daten in einem entfernten Rechenzentrum (umgangssprachlich: "`Ich lade das Bild mal in die \Gls{glos:cloudLabel} hoch."'), aber auch die Ausführung von Programmen, die nicht auf dem lokalen Rechner installiert sind, sondern eben in der (metaphorischen) Wolke (englisch \textit{cloud}). \textit{- Quelle: Wikipedia} \footcite{Cloud_Computing__Wikipedia_2014-10-03}
\end{quote}

Oder ganz einfach:

\begin{quote}
	In simple terms, using \textit{cloud} computing means storing your files on a place that is not your local hard drive.  \textit{- Quelle: news.com.au} \footcite{What_really_is_The_Cloud?_And_how_does_it_work?_A_simple_explainer_2014-10-31}
\end{quote}
\begin{quote}
	Einfach gesagt, die \textit{Cloud} bezeichnet das Speichern von Daten die nicht auf der lokalen Festplatte abgelegt werden, sondern irgendwo anders. (sinngemässe Übersetzung des oberen)
\end{quote}
% evt. was für die Präsi: https://www.youtube.com/watch?v=ecZL4Q2EVuY

So wird ersichtlich, dass keine eindeutige Definition vom Begriff \textit{\Gls{glos:cloudLabel}} existiert.

Im Rahmen unserer Arbeit haben wir selbst eine Definition von \Gls{glos:cloudLabel} erstellt:
\begin{quote}
	Cloud bietet dem Benutzer Software oder Hardware zur Benutzung an, ohne das jener im Detail weiss, auf was für Infrastruktur er arbeitet.
	Meist stehen Komponenten im Internet bereit, so dass weltweiter Zugriff ermöglicht wird.
\end{quote}

Beispiele für \textit{\Gls{glos:cloudLabel} Computing}:
\begin{itemize}
	\item Dropbox
	\item Google Mail GMail
	\item Google Drive
	\item GitHub
\end{itemize}

\section{Herkunft}
Das Konzept von skalierbaren (Mainframe-) Computern wurde schon in den 60er Jahren definiert.
Wer dafür den Begriff \textit{\Gls{glos:cloudLabel}} erfunden hat ist unklar. Die Bezeichnung selbst ist kein neuer Begriff,
sondern wird allgemein für grosse Ansammlungen von Objekten gebraucht, bei denen man aus der Distanz betrachtet nicht auf Details eingehen will.\footcite{Cloud_computing_-_Wikipedia_the_free_encyclopedia_2014-10-31}
Als Symbol für das Internet wird die Wolke seit 1994 verwendet.

Massgeblich beigetragen zur Nutzung des Begriffes haben zwei bekannte Unternehmen im Jahr 2006.

Einerseits hatte \textit{Amazon} durch den Tag eine zehnfach höhere Spitzenlast als in der Nacht. Als Konsequenz entwickelten sie skalierbare \Gls{glos:cloudLabel}-Lösungen.
Diese entwickelte Lösungen boten sie als Produkt \textit{Elastic Compute Cloud} öffentlich an. \footcite{Announcing_Amazon_Elastic_Compute_Cloud_(Amazon_EC2)_-_beta_2014-10-31}

Auch \textit{Google} CEO Eric Schmidt hatte zur selben Zeit folgende Aussage gemacht: \footcite{Who_Coined_'Cloud_Computing'?_|_MIT_Technology_Review_2014-10-31}
\begin{quote}
	"`What’s interesting [now] is that there is an emergent new model.
	I don’t think people have really understood how big this opportunity really is.
	It starts with the premise that the data services and architecture should be on servers.
	We call it cloud computing - they should be in a "`cloud"' somewhere."'
\end{quote}


\section{Services für Workstation}
% spezifisch für unser Fall: Workstation
Mittlerweile gibt es viele verschiedene Möglichkeiten und Anbieter, bei welchen man seine Daten in der \Gls{glos:cloudLabel} ablegen kann. Mit \Gls{iaasLabel} wird die Idee ein Level weiter getragen.
Anstelle des Kaufes von Infrastrukturen wie Personal Computer oder Servern, werden diese bei Bedarf gemietet.

Konkret gibt es auch den Begriff \Gls{daasLabel}, der bezeichnet exakt die Bereitstellung einer Workstation in der \Gls{glos:cloudLabel}.

\subsection{Vorteile}
\label{sec:Vorteile}
\subsubsection{Belastungsspitzen}
Bei vielen Anwendungen gibt es tägliche, saisonale, jährliche oder andere Belastungsspitzen, zu welcher zusätzliche Leistung benötigt wird.
Traditionell müsste dann zu jeder Zeit die Leistung zur Verfügung stehen, um die Spitzen abzufangen. Wenn weniger Betrieb herrscht liegt dieses Potenzial brach.

Mit \Gls{iaasLabel} / \Gls{daasLabel} kann dynamisch zu Lastzeiten mehr Leistung hinzu gemietet werden.

% \subsubsection{Bring your own Device}
% mha: nimmt viele Security issues ab, die bei byod vorhanden sind. -> mha: das haben wir aber abgegrenzt.


\subsubsection{Ausfallsicherheit}
Um eine annähernde Ausfallsicherheit von 100\% zu erreichen, braucht es redundante Hardware, wodurch die Kosten für die Anschaffung und den Betrieb rasch steigen.

\subsubsection{Skalierbarkeit}
Da der Erfolg einer Firma nicht im Voraus bekannt ist, bietet \Gls{iaasLabel} den Vorteil, dass man immer nur genau die Infrastruktur zahlt, die man auch benötigt.
Innerhalb kurzer Frist kann man Leistung und verbundene Kosten steigern oder verkleinern.
Das Restrisiko, dass ungenutzte Hardware bezahlt, aber nicht verwendet wurde sinkt.

\subsubsection{Kostenübersicht}
Die Kosten können für das Business kalkuliert werden, da diese voraussehbar sind.
Es müssen keine Wartungen, Ersatzteile oder ähnliches in das Budget mit eingerechnet werden.

% neuer Cloud norm
% http://www.computerwoche.de/a/die-neue-iso-iec-27018-im-ueberblick,3069892

% \section{Vorliegendes Material}

% \subsection{Angebotsübersicht (Preise für Cloud Lösungen)}
