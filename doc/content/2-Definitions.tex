\chapter{Allgemein zur Cloud}

\section{Definitionen}

% allgemeine Definition von Cloud

Der Begriff \textit{Cloud} resp. \textit{Cloud Computing} wird sehr häufig gebraucht (eine Google Suche findet 136 Mio. Resultate zum zweit genannten Begriff).
Trotz dem vielen Wortgebrauch ist eine klare Definition kaum bekannt.

Folgend drei Beispiele, die mögliche Definitionen aufzeigen sollen.

\begin{quote}
	Die \textit{Cloud} ist eines der ältesten Sinnbilder der Informationstechnik und steht als solches für Rechnernetze, deren Inneres unbedeutend oder unbekannt ist. \textit{- Quelle: The Wall Street Journal}  \footcite{The_Internet_Industry_Is_on_a_Cloud_--_Whatever_That_May_Mean_-_WSJ_2014-10-03}
\end{quote}

Oder ein wenig konkreter:

\begin{quote}
	Unter \textit{Cloud Computing} (deutsch etwa: Rechnen in der Wolke) versteht man das Speichern von Daten in einem entfernten Rechenzentrum (umgangssprachlich: "`Ich lade das Bild mal in die Cloud hoch."'), aber auch die Ausführung von Programmen, die nicht auf dem lokalen Rechner installiert sind, sondern eben in der (metaphorischen) Wolke (englisch \textit{cloud}). \textit{- Quelle: Wikipedia} \footcite{Cloud_Computing__Wikipedia_2014-10-03}
\end{quote}

Oder ganz einfach:

\begin{quote}
	In simple terms, using \textit{cloud} computing means storing your files on a place that is not your local hard drive.  \textit{- Quelle: news.com.au} \footcite{What_really_is_The_Cloud?_And_how_does_it_work?_A_simple_explainer_2014-10-31}
\end{quote}
\begin{quote}
	Einfach gesagt, die \textit{Cloud} bezeichnet das Speichern von Daten die nicht auf der lokalen Festplatte abgelegt werden, sondern irgendwo anders. (sinngemässe Übersetzung des oberen)
\end{quote}
% evt. was für die Präsi: https://www.youtube.com/watch?v=ecZL4Q2EVuY

So wird ersichtlich, dass keine eindeutige Definition vom Begriff \textit{Cloud} existiert.

Im Rahmen unserer Arbeit haben wir selbst eine Definition von Cloud erstellt:
%todo: quote Cloud
\begin{quote}
	Wichtige Punkte:
	- Wolke: da nichts konkret auf einem Computer läuft -> sondern auf skalierbaren Infrastrukturen -> Wolke zeigt diese Ungewissheit wo und wie was abläuft (schwammiger Umriss)
	- von überall via Internet erreichbar
	- jegliche Service: Mail, File, Software,...
	- 
	
\end{quote}

Beispiele für \textit{Cloud Computing}:
\begin{itemize}
	\item Dropbox
	\item Google Mail GMail
	\item Google Drive
	\item GitHub
\end{itemize}

\section{Herkunft}
Das Konzept von skalierbaren (Mainframe-) Computern wurde schon in den 60er Jahren definiert.
Wer dafür den Begriff \textit{Cloud} erfunden hat ist unklar. Die Bezeichnung selbst ist kein neuer Begriff,
sondern wird allgemein für grosse Ansammlungen von Objekten gebraucht, bei denen man aus der Distanz betrachtet nicht auf Details eingehen will.\footcite{Cloud_computing_-_Wikipedia_the_free_encyclopedia_2014-10-31}
Als Symbol für das Internet wird die Wolke seit 1994 verwendet.

Massgeblich beigetragen zur Nutzung des Begriffes haben zwei bekannte Unternehmen im Jahr 2006.

Einerseits hatte \textit{Amazon} durch den Tag eine zehnfach höhere Spitzenlast als in der Nacht. Als Konsequenz entwickelten sie skalierbare Cloud-Lösungen.
Diese entwickelte Lösungen boten sie als Produkt \textit{Elastic Compute Cloud} öffentlich an. \footcite{Announcing_Amazon_Elastic_Compute_Cloud_(Amazon_EC2)_-_beta_2014-10-31}

Auch \textit{Google} CEO Eric Schmidt hatte zur selben Zeit folgende Aussage gemacht: \footcite{Who_Coined_'Cloud_Computing'?_|_MIT_Technology_Review_2014-10-31}
\begin{quote}
	"`What’s interesting [now] is that there is an emergent new model.
	I don’t think people have really understood how big this opportunity really is.
	It starts with the premise that the data services and architecture should be on servers.
	We call it cloud computing - they should be in a "`cloud"' somewhere."'
\end{quote}


\section{Platform-as-a-Service (PaaS): Workstation}
% spezifisch für unser Fall: Workstation



\section{Vorliegendes Material}

\subsection{Angebotsübersicht (Preise für Cloud Lösungen)}
