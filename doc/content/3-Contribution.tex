
% Eigener Beitrag: Beschreibung, Begründung, Aufzeigung, Methode, Fazit

\chapter{Analyse}

% ...
\section{Allgemein}
% Amazon Lösung: https://aws.amazon.com/marketplace/pp/B00FGB3528
	% https://aws.amazon.com/marketplace/pp/B00GDZINWI
% Free VPS https://manage.haphost.com
% VmWare: http://www.vmware.com/products/workstation
% Microsoft: http://www.microsoft.com/en-us/server-cloud/products/virtual-desktop-infrastructure/default.aspx
	% http://www.microsoft.com/en-us/windows/enterprise/products-and-technologies/virtualization/default.aspx
	% Microsoft Azure
	% Microsoft HyperV: http://www.microsoft.com/en-us/server-cloud/solutions/virtualization.aspx
	
Im Zentrum der Arbeit werden die verschiedenen Arten, wie man einen Arbeitsplatz in den Cloud Hosten kann beschrieben. Von den beiden Typen wird jeweils ein Vertreter aus der Marktwirtschaft analysiert.

\section{Typen}
% Crosslink zu Vorteile einfügen
Um die Arbeitsplätze einer Firma in die Cloud zu heben, muss zuerst entschieden werden, ob man diese in einem eigenen Rechenzentrum betreiben möchte, oder ob man diese Arbeit an eine externe Firma auslagert. 

Beide Vorgehensweisen beinhalten Vorteile und Tücken. Allgemeine Vorteile welche beide Arten mit sich bringen sind bereits unter dem Kapitel Vorteile beschrieben. Crosslink: \ref{sec:Vorteile}

Wird das Hosting ausgelagert, hat man eine monatliche oder jährliche Abrechnung. Dadurch werden die Kosten kalkulierbar. Auch fallen grosse initiale Anschaffungskosten weg.

Firmen mit gesetzlichen Auflagen wie Versicherungen, Banken, etc. kann es dabei Probleme mit dem Datenschutz geben. Probleme können sich ergeben wenn die Daten im Ausland gelagert werden, oder sie nicht konsequent verschlüsselt übertragen und gespeichert werden.

Bei solchen Problem bietet sich ein eigens Hosting an. Die Anschaffungskosten sind massiv höher und die Betriebskosten können volatil sein, man hat jedoch die volle Kontrolle wo die Daten gelagert werden und ob die Kommunikationskanäle ausreichend geschützt sind.

Diese Arbeit beschäftigt sich näher mit Amazon WorkSpaces, als Vertreter der ausgelagerten Lösung, sowie VmWare Virtual Desktop Manager, mit der man die Arbeitsplätze auf den eigenen Servern betreiben kann.

\section{Amazon WorkSpaces}
% mh: http://aws.amazon.com/de/workspaces
% http://en.wikipedia.org/wiki/Amazon.com
% http://aws.typepad.com/aws/2013/11/tco-comparison-amazon-workspaces-and-traditional-virtual-desktop-infrastructure-vdi.html

Amazon startete als einfache Bücherverkaufsplatform, fügte jedoch bald weitere Artikel wie CDs, MP3s, Software etc. hinzu, wie auch eigene Produkte wie Tablets und E-Books.
Zudem wird einem eine breite Platform verschiedener Cloud Computing Services, \Gls{awsLabel} genannt, angeboten.\footcite{Amazon.com_-_Wikipedia_the_free_encyclopedia_2014-11-15}

Die Lösung für Workstations in der Cloud heisst \textit{Amazon WorkSpaces}\footcite{AWS_Amazon_WorkSpaces_2014-11-03}.
Mit Amazon WorkSpaces wird einem eine vollständige \Gls{vdiLabel} angeboten.

Leider können WorkSpaces nicht gratis getestet werden. Aus diesem Grund gibt es hier keine Tests und Erfahrungsberichten.

\subsection{Angebot}
Ab 35 USD pro Monat wird eine \Gls{vdiLabel} angeboten. Auf diese kann per iPad, Android, Kindle Fire, PC und Mac zugegriffen werden.\\
Zur Auswahl stehen zwei Typen: \textit{Standard} und \textit{Leistung}.

\begin{table}[H]
	\centering
	\small\renewcommand{\arraystretch}{1.4}  
	\rowcolors{1}{tablerowcolor}{tablebodycolor}
	%
	\captionabove[Amazon WorkSpace-Angebote]{Amazon WorkSpaces Angebots-Übersicht}
	%
	\begin{tabularx}{0.9\textwidth}{X | Y | Y | Y | Y }
		\hline
		\rowcolor{tableheadcolor}
		\textbf{WorkSpaces-Paket} & \textbf{vCPU}\footcite{Virtual_CPUs_with_Amazon_Web_Services_2014-11-15} & \textbf{RAM} & \textbf{Benutzerspeicher} & \textbf{Gebühren}\footnote{Preise entsprechen dem günstigsten Angebot, je nach Standort können zusätzlich noch bis zu 18 USD anfallen.}\\
		\rowcolor{tableheadcolor}
		 & \# & GiB & GB & USD/mtl.\\
		\hline
			\textbf{Standard} & 1 & 3.75 & 50 & 35\\
			\textbf{Leistung} & 2 & 7.5 & 100 & 60\\
		\hline
	\end{tabularx}
\end{table}

Folgende Standard Software ist vorinstalliert:
\begin{itemize}
	\item Adobe Reader
	\item Internet Explorer 9
	\item Firefox
	\item 7-Zip
	\item Adobe Flash
\end{itemize}

Für zusätzliche 15 USD/Monat kriegt man weitere Software (inkl. Lizenzen) im "`Plus"'-Paket:
\begin{itemize}
	\item Microsoft Office Professional 2010
	\item Trend Micro Worry-Free Business Security Services
	\item WinZip
\end{itemize}

Als Standord der Hostingcentern kann zwischen folgenden fünf \Gls{awsLabel}-Regionen ausgewählt werden. Die Preise varieren je nach Standort.
\begin{itemize}
	\item USA Ost (Nord-Virginia)
	\item USA West (Oregon)
	\item EU (Irland)
	\item Asien-Pazifik (Sydney)
	\item Asien-Pazifik (Tokio)
\end{itemize}


\subsection{Aufwand}
Der Initial- und Wartungsaufwand (zeitlicher Aufwand, wie auch das Vorhanden sein des technisches Verständnisses) sind deutlich geringer, als wenn auf andere Weise eine gleichwertige Lösung angeboten werden soll.\footnote{Auf Vor- und Nachteile bezüglich \textit{Bring your own Device} wird in diesem Dokument nicht eingegangen.}
Es muss lediglich eine Internetverbindung angeboten werden.
Das Management der \Gls{vdiLabel} ist deutlich einfacher, da keine Anforderungen an Server Hardware, internes Netzwerk, Backup Lösung und Server Umgebung gestellt werden.


\subsection{Kosten}
Die Kosten einer VDI variieren je nach \Gls{awsLabel}-Region.\footcite{AWS_Amazon_WorkSpaces_Preise_2014-11-15}

\begin{table}[H]
	\centering
	\small\renewcommand{\arraystretch}{1.4}  
	\rowcolors{1}{tablerowcolor}{tablebodycolor}
	%
	\captionabove[Amazon WorkSpaces Preise]{Amazon WorkSpaces Preis-Übersicht nach Region}
	%
	\begin{tabularx}{0.9\textwidth}{X | Y | Y | Y | Y | Y}
		\hline
		\rowcolor{tableheadcolor}
		\textbf{Paket} & \textbf{Nord-Virginia} & \textbf{Oregon} & \textbf{Irland} & \textbf{Sydney} & \textbf{Tokio}\\
		\hline
		\textbf{Standard} & 35 & 35 & \textbf{37} & 45 & 47\\
		\textbf{Leistung} & 60 & 60 & \textbf{64} & 75 & 78\\
		\hline
		\multicolumn{6}{r}{
			Bemerkungen: alle Preise in USD, besucht am 17. 11. 2014
		}
	\end{tabularx}
\end{table}

Wie bereits erwähnt, kann das Software Paket "`Plus"' für zusätzliche 15 USD pro Monat beansprucht werden.

Um eine Kosteneinsparung gegenüber einer eigener \Gls{vdiLabel} auszurechnen, bietet Amazon eine Excel-Vorlage.\footcite{TCO_Comparison_Amazon_WorkSpaces_and_Traditional_Virtual_Desktop_Infrastructure_VDI_2014-11-15}

\subsection{Sicherheit und Geschwindigkeit}
Beim Thema Sicherheit ist einiges machbar.
Es können Richtlinien aus dem \Gls{adLabel} übernommen werden.
Auf dem Client selbst werden keine Daten gespeichert.
Amazone bietet eine Ausfallsicherheit von 99,999999999 \% an.
Ein automatisches Backup der Benutzerdaten erfolgt alle 12 Stunden.

Für die Bildschirmübertragung (betrifft nur die effektiven Pixeln) werden die Daten komprimiert und verschlüsselt.
Nebst den Bildschirmdaten verlassen keinerlei Daten die \Gls{awsLabel} Server Infrastruktur.
Durch die Komprimierung ist eine hohe Auflösung der Bildschirminhalte möglich. Selbst bei einer langsamer Internetverbindung genügt das Bild noch zum Arbeiten.

\subsection{weitere Features}
\begin{itemize}
	\item Das Interface ist für mobile Nutzung optimiert.
	\item Bei Bereitstellung eines \Gls{vdiLabel} kann automatisch ein Mail mit den nötigen Schritten an den Enduser generiert werden.
	\item Für die Authentifizierung kann eine Multi-Factor Methode verwendet werden.
	\item Amazon selbst stellt den \textit{Amazon Zocalo Sync} zur Verfügung (entspricht der ähnlichen Funktion wie der Dropbox-Service\footcite{Dropbox_2014-11-15}).
	\item Die Verwendungen von lokalen Druckern ist möglich.
%	\item Der Client ist momentan nur in englischer Sprache verfügbar.
%	\item Zahlung erfolgt pro Monat und kann monatlich angepasst werden. Man muss grundsätzlich nicht für etwas zahlen, dass gar nicht verwendet wird.
	\item Gerade für temporäre Arbeitnehmer können sehr einfach Maschinen für die Dauer gemietet werden und anschliessend kann der Zugriff wieder entfernt werden.
\end{itemize}



\section{Microsoft Hyper-V}
% sl
\subsection{Aufwand}

\subsection{Kosten}

\subsection{Sicherheit}

\subsection{Geschwindigkeit}

\subsection{Produkt Lösungen}



\section{VmWare Virtual Desktop Manager}
% sl
% Quelle: http://www.vmware.com/pdf/vdm20_intro.pdf
VmWare ist eine amerikanische Firma mit Sitz im Palo Alto, Californien. Gegründet wurde die Organisation 1998 und ging 2007 an die Börse.
Als Hauptproduct bietet VmWare Virtualisierungssoftware für Server Infrastrukturen an. Ein Teil davon, welchem sich dieses Kapitel gewidmet, ist die on-premis VDI Lösung \Gls{vdmLabel}, welche auf dem Kern von VmWare, der VMware Infrastructure 3 aufbaut.

\subsection{Aufwand}
Der Initialaufwand für die Verwendung von \Gls{vdmLabel} ist relativ hoch, da zuvor die ganze Server Infrastruktur mit VmWare Produkten aufgesetzt werden muss.
Im Gegenzug gibt es einige Vorteile daraus:
\begin{itemize}
	\item Kontrolle und Management der Desktops aus einer Applikation heraus.
	\item Die Benutzer merken nichts von der Umstellung.
	\item Geringere Kosten als eine on-premis Lösung.
\end{itemize}


\subsection{Kosten}

\subsection{Sicherheit}

\subsection{Geschwindigkeit}

\subsection{Produkt Lösungen}





