
% Eigener Beitrag: Beschreibung, Begründung, Aufzeigung, Methode, Fazit

\chapter{Analyse}

% ...
\section{Allgemein}
% Amazon Lösung: https://aws.amazon.com/marketplace/pp/B00FGB3528
	% https://aws.amazon.com/marketplace/pp/B00GDZINWI
% Free VPS https://manage.haphost.com
% VmWare: http://www.vmware.com/products/workstation
% Microsoft: http://www.microsoft.com/en-us/server-cloud/products/virtual-desktop-infrastructure/default.aspx
	% http://www.microsoft.com/en-us/windows/enterprise/products-and-technologies/virtualization/default.aspx
	% Microsoft Azure
	% Microsoft HyperV: http://www.microsoft.com/en-us/server-cloud/solutions/virtualization.aspx

\section{Typen}
% insource/outsource

\section{Amazon Workspaces}
% mh: http://aws.amazon.com/de/workspaces
% http://en.wikipedia.org/wiki/Amazon.com
% http://aws.typepad.com/aws/2013/11/tco-comparison-amazon-workspaces-and-traditional-virtual-desktop-infrastructure-vdi.html

Amazon startete als einfache Bücherverkaufsplatform, fügte jedoch bald weitere Artikel wie CDs MP3, Software etc. hinzu, wie auch eigene Produkte wie Tablets und E-Books.
Zudem wird einem eine breite Platform jeglicher Cloud Computing Services - \Gls{awsLabel} - angeboten.\footcite{Amazon.com_-_Wikipedia_the_free_encyclopedia_2014-11-15}
\textit{Amazon Workspaces}\footcite{AWS_Amazon_WorkSpaces_2014-11-03} heisst das Angebot für Workstation Angebote in der Cloud.

Mit Amazon Workspaces wird einem eine vollständige \Gls{vdiLabel} angeboten. Sprich eine Workstation in der Cloud speziell geeignet für Firmen.

Leider ist es nicht möglich Workplaces zu gratis zu testen. Aus diesem Grund gibt es hier keine Testberichte.

\subsection{Angebot}
Ab 35 USD pro Monat wird eine \Gls{vdiLabel} angeboten. Auf diesen kann per iPad, Android, Kindle Fire, PC und Mac zugegriffen werden.\\
Zur Auswahl stehen zwei Typen: \textit{Standard} und \textit{Leistung}.


%\begin{table}[H]
%	\centering
%%	\small\renewcommand{\arraystretch}{1.4}  
%	\rowcolors{1}{tablerowcolor}{tablebodycolor}
%	%
%	\captionabove[Amazon Workplace Angebote]{Amazon Workplaces Angebots-Übersicht}
%	%
%	\begin{tabular}{l | r | r @{.} l | r | r }
%		\hline
%		\rowcolor{tableheadcolor}
%		\textbf{Bezeichnung} & \textbf{vCPU}\footcite{Virtual_CPUs_with_Amazon_Web_Services_2014-11-15} & \multicolumn{2}{c|}{\textbf{RAM}} & \textbf{Benutzerspeicher} & \textbf{Gebühren}\footnote{Preise entsprechen dem günstigsten Angebot, je nach Standort können zusätzlich noch bis zu 18 USD anfallen.}\\
%		\rowcolor{tableheadcolor}
%		 & \# & \multicolumn{2}{c|}{GiB} & GB & USD/Monat\\
%		\hline
%			\textbf{Standard} & 1 & 3 & 75 & 50 & 35\\
%			\textbf{Leistung} & 2 & 7 & 5 & 100 & 60\\
%		\hline
%	\end{tabular}
%\end{table}

\begin{table}[H]
	\centering
	\small\renewcommand{\arraystretch}{1.4}  
	\rowcolors{1}{tablerowcolor}{tablebodycolor}
	%
	\captionabove[Amazon Workplace Angebote]{Amazon Workplaces Angebots-Übersicht}
	%
	\begin{tabularx}{0.8\textwidth}{X | Y | Y | Y | Y }
		\hline
		\rowcolor{tableheadcolor}
		\textbf{Bezeichnung} & \textbf{vCPU}\footcite{Virtual_CPUs_with_Amazon_Web_Services_2014-11-15} & \textbf{RAM} & \textbf{Benutzerspeicher} & \textbf{Gebühren}\footnote{Preise entsprechen dem günstigsten Angebot, je nach Standort können zusätzlich noch bis zu 18 USD anfallen.}\\
		\rowcolor{tableheadcolor}
		 & \# & GiB & GB & USD/mtl.\\
		\hline
			\textbf{Standard} & 1 & 3.75 & 50 & 35\\
			\textbf{Leistung} & 2 & 7.5 & 100 & 60\\
		\hline
	\end{tabularx}
\end{table}

Folgende Standard Software ist vorinstalliert:
\begin{itemize}
	\item Adobe Reader
	\item Internet Explorer 9
	\item Firefox
	\item 7-Zip
	\item Adobe Flash
\end{itemize}

Für zusätzliche 15 USD/Monat kriegt man weitere Software (inkl. Lizenzen):
\begin{itemize}
	\item Microsoft Office Professional 2010
	\item Trend Micro Worry-Free Business Security Services
	\item WinZip
\end{itemize}

Als Standord der Hostingcentern kann zwischen folgenden fünf AWS Regionen ausgewählt werden. Die Preise varieren je nach Standort.
\begin{itemize}
	\item USA Ost (Nord-Virginia)
	\item USA West (Oregon)
	\item EU (Irland)
	\item Asien-Pazifik (Sydney)
	\item Asien-Pazifik (Tokio)
\end{itemize}


\subsection{Aufwand}
Der Initial- und Wartungsaufwand (zeitlicher Aufwand, wie auch das Vorhanden sein des technisches Verständnis) sind deutlich geringer, als wenn auf andere Weise eine gleichwertige Lösung angeboten werden soll. \footnote{Auf Vor- und Nachteile bezüglich \textit{Bring your own Device} wird in diesem Dokument nicht eingegangen.}
Es muss lediglich eine Internetverbindung angeboten werden.
Das Management der \Gls{vdiLabel} ist deutlich einfacher, da keine Anforderungen an Server Hardware, internes Netzwerk, Backup Lösung und Server Umgebung gestellt werden.


\subsection{Kosten}
%todo


\footcite{AWS_Amazon_WorkSpaces_Preise_2014-11-15}
\footcite{TCO_Comparison_Amazon_WorkSpaces_and_Traditional_Virtual_Desktop_Infrastructure_VDI_2014-11-15}


\subsection{Sicherheit und Geschwindigkeit}
Beim Thema Sicherheit ist einiges machbar.
Es können Richtlinien aus dem \Gls{adLabel} übernommen werden.
Auf dem Client selbst werden keine Daten gespeichert.
Amazone bietet eine Ausfallsicherheit von 99,999999999 \% an.
Ein automatisches Backup der Benutzerdaten erfolgt alle 12 Stunden.

Für die Bildschirmübertragung (betrifft nur die effektiven Pixeln) werden die Daten komprimiert und verschlüsselt.
Nebst den Bildschirmdaten verlassen keinerlei Daten in irgendeiner Form die \Gls{awsLabel} Server Infrastruktur.
Durch die Komprimierung ist eine hohe Auflösung der Bildschirminhalte möglich. Selbst bei einer langsamer Internetverbindung genügt das Bild noch zum Arbeiten.

\subsection{weitere Features}
\begin{itemize}
	\item Das Interface ist Mobile optimiert.
	\item Bei Bereitstellung eines \Gls{vdiLabel} kann automatisch ein Mail mit den nötigen Schritten an den Enduser generiert werden.
	\item Für die Authentifizierung kann eine Multi-Factor Methode verwendet werden.
	\item Amazon selbst stellt den \textit{Amazon Zocalo Sync} zur Verfügung (entspricht der ähnlichen Funktion wie der Dropbox-Service\footcite{Dropbox_2014-11-15}).
	\item Die Verwendungen von lokalen Druckern ist möglich.
	\item Der Client ist momentan nur in englischer Sprache verfügbar.
	\item Zahlung erfolgt pro Monat und kann monatlich angepasst werden. Man muss grundsätzlich nicht für etwas zahlen, dass gar nicht verwendet wird.
	\item Gerade für temporäre Arbeitnehmer können sehr einfach Maschinen für die Dauer gemietet werden und anschliessend kann der Zugriff wieder entfernt werden.
\end{itemize}





\section{"'VPS Haphost"' oder doch: "`On-Premise Vergleich"'}
% mh: da keine Rückmeldung, evt. Vergleich zu On-Premise Lösung
\subsection{Aufwand}

\subsection{Kosten}

\subsection{Sicherheit}

\subsection{Geschwindigkeit}

\subsection{Produkt Lösungen}



\section{Microsoft Hyper-V}
% sl
\subsection{Aufwand}

\subsection{Kosten}

\subsection{Sicherheit}

\subsection{Geschwindigkeit}

\subsection{Produkt Lösungen}



\section{VmWare}
% sl
\subsection{Aufwand}

\subsection{Kosten}

\subsection{Sicherheit}

\subsection{Geschwindigkeit}

\subsection{Produkt Lösungen}





