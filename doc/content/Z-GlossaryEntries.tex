% !TeX encoding=utf8
% !TeX spellcheck = de_CH_frami

%%% --- Acronym definitions
\IfDefined{newacronym}{%
% examples

% our used acronyms
\newacronym{vdiLabel}{VDI}{Virtual Desktop Infrastructure}
\newacronym{adLabel}{AD}{Active Directory}
\newacronym{awsLabel}{AWS}{Amazon Web Services}
\newacronym{iaasLabel}{IaaS}{Infastructure-as-a-Service}
\newacronym{daasLabel}{DaaS}{Desktop-as-a-Service}
\newacronym{vdmLabel}{VDM}{Virtual Desktop Manager}
\newacronym{dmzLabel}{DMZ}{Demilitarized Zone}
\newacronym{ldapLabel}{LDAP}{Lightweight Directory Access Protocol}
\newacronym{httpsLabel}{HTTPS}{Hypertext Transfer Protocol Secure}
\newacronym{ssdLabel}{SSD}{Solid State Drive}
\newacronym{cpuLabel}{CPU}{Central Processing Unit}
\newacronym{pueLabel}{PUE}{Power usage effectiveness}
\newacronym{b2bLabel}{B2B}{Business-To-Business}
}
% use it with \Gls{vdiLabel}

%%% --- Symbol list entries

%\newglossaryentry{symb:Pi}{%
%  name=$\pi$,%
%  description={mathematical constant},%
%  sort=symbolpi, type=symbolslist%
%}


%%% --- Glossary entries
\newglossaryentry{glos:dmzLabel}{name=DMZ,
  description={Die DMZ oder Demilitarized Zone ist ein logisches oder physikaliches Subnetzwerk, welches interne Server zu einem grösseren, nicht vertrauenswürdigen Netzwerk verbindet. Es bietet eine zusätzliche Schicht Sicherheit und gibt den Administratoren mehr Kontrolle darüber, wer Zugriff auf Netzwerkressourcen hat.}
}
\newglossaryentry{glos:ldapLabel}{name=LDAP,
  description={Das Lightweight Directory Access Protocol ist ein offenes, Anbieter unspezifisches Applikations-Protokoll um ein verteilter Verzeichnissdienst über das Internet Protokoll bereit zu stellen. Darüber können Informationen über Benutzer, Systeme, Netzwerke, Services und Applikationen abgefragt werden.}
}
\newglossaryentry{glos:httpsLabel}{name=HTTPS,
  description={Das Hypertext Transfer Protocol Secure ist ein Kommunikations Protokoll für eine verschlüsselte Kommunikation über ein Compunternetzwerk.}
}
\newglossaryentry{glos:pueLabel}{name=PUE,
  description={Mit der Power usage effectiveness wird der Stromverbrauch eines Datenzenters berechnet. Dieser setzt sich folgendermassen zusammen: 
  \begin{gather*}
  PUE = \frac{Total\: Stromkosten}{Stromkosten\: der\: IT}
  \end{gather*}}
}
\newglossaryentry{glos:loadBalancingLabel}{name=Load Balancing,
  description={Load Balancing verteilt die Arbeitsbelastung auf verschiedene Systeme. Damit kann die Antwortzeit reduziert werden. Wenn ein System ausfällt hat es immer noch weitere die funktionieren. Dies steigert die Ausfallsicherheit des Gesamtsystems.}
}
\newglossaryentry{glos:thinClientLabel}{
  name=Thin Client,
  plural={Thin Clients},
  description={Ein Thin Client ist ein günstiger, rechen-schwacher Computer. Er wird dazu verwendet, um arbeiten zu erledigen die auf einem rechen-starken Server statt findet.}
}
\newglossaryentry{glos:homeOfficeLabel}{
  name=Home Office,
  description={Mit Home Office ist das Arbeiten von zuhause gemeint. Dabei wird eine sichere Verbindung von Zuhause auf die Infrastruktur der Firma erzeugt.}
}
\newglossaryentry{glos:vpnLabel}{
  name=Virtual Private Netzwork,
  description={Ein Virtual Private Netzwork wird zwischen einem Teilnehmer und dem Server aufgebaut. Dabei wird innerhalb einen privaten Netzwerkes, wie dem Internet, ein privates und sicheres Netzwerk erstellt.s}
}
% use it with \gls{glos:DVD}
% use plural with \glspl{thinClientLabel}

