% !TeX encoding=utf8
% !TeX spellcheck = de_CH_frami

%%% --- Acronym definitions
\IfDefined{newacronym}{%
% examples
\newacronym{MFD}{MFD}{mode field diameter}
\newacronym{CPA}{CPA}{chirped pulse amplification}
\newacronym{NA}{NA}{numerical apertur}
\newacronym{MMI}{MMI}{multi-mode interference}
\newacronym{SLM}{SLM}{spatial light modulator}
\newacronym{LCD}{LCD}{liquid crystal display}
\newacronym{px}{px}{Pixel}
\newacronym{DNA}{DNA}{deoxyribonucleic acid}
\newacronym{DOF}{DOF}{depth of focus}
\newacronym{PSF}{PSF}{point spread function}
\newacronym{SNOM}{SNOM}{scanning nearfield optical microscope}
\newacronym{FWHM}{FWHM}{full width at half maximum}
\newacronym[longplural=Frames per Second]{fpsLabel}{FPS}{Frame per Second}

% our used acronyms
\newacronym{vdiLabel}{VDI}{Virtual Desktop Infrastructure}
\newacronym{adLabel}{AD}{Active Directory}
\newacronym{awsLabel}{AWS}{Amazon Web Services}

}%
% use it with \Gls{vdiLabel}

%%% --- Symbol list entries

%\newglossaryentry{symb:Pi}{%
%  name=$\pi$,%
%  description={mathematical constant},%
%  sort=symbolpi, type=symbolslist%
%}


%%% --- Glossary entries

\newglossaryentry{glos:DVD}{name=DVD,
  description={DVD is an optical disc storage media format, invented and
  developed by Philips, Sony, Toshiba, and Panasonic in 1995. DVDs offer
  higher storage capacity than Compact Discs while having the same dimensions.
  The basis of the DVD name stems from the term \textit{digital versatile disc}. (Source: wikipedia)}
}
% use it with \gls{glos:DVD}

