\chapter{Beschreibung der Aufgabe}

\section{Aufgabenstellung}
% aus Aufgabenstellung v2:
% (https://docs.google.com/document/d/1RX4SnkgstYHtR3L8P-43zTOZWYBdmZ10h4wnGvS7nMg/edit?usp=sharing)


\subsection{Ausgangslage}
Die meisten Menschen arbeiten gegenwärtig mit einer traditionellen "`\gls{glos:onPremiseLabel}"' Computerlösung.
Ein Grossteil der Rechenleistung bleibt ungenutzt, da nur zu Peaks die volle Leistung benötigt wird.
Spart man Geld bei den Workstations, muss der Benutzer warten, sobald rechen-intensive Arbeiten durchgeführt werden.

Mit einer \Gls{glos:cloudLabel}-Lösung soll es möglich sein, dem User ständig so viel Rechenpower bereitzustellen, wie er benötigt.
Dies spart gesamt Ressourcen ein, da die Leistung zwischen den Benutzern geteilt werden kann.

\subsection{Ziele der Arbeit}
\label{sec:desc:targets}
Die geschilderte Ausgangslage drängt den Gedanken an eine “soziale” Ressourcen-Aufteilung auf.

Auf folgende Fragen soll eingegangen werden:
\begin{itemize}
	\item Können, in Anbetracht von Kosten und Aufwand, verfügbare Ressourcen effizient, aber trotzdem gerecht, zur Verfügung gestellt werden?
	\item Was sind die Vor- und Nachteile einer \Gls{glos:cloudLabel}-Lösung?
	\item Welche Software für solche Ansprüche ist momentan auf dem Markt führend?
	\item Wie kann eine Workstation in der \Gls{glos:cloudLabel} angeboten werden?
%	\item Welche Hardware wird für die Bereitstellung der \Gls{glos:cloudLabel} benötigt?
\end{itemize}


\subsection{Eingrenzung}
% - Abklärung Hosting von Working Station

Oben erwähnte Fragen im \cref{sec:desc:targets} sollen beantwortet werden.
Dieses Doukment wurde so formuliert, dass technische Personen die Aussagen nachvollziehen und verstehen können.

Es wird auf konkrete Lösungen für \Gls{glos:cloudLabel}-Computing der Anbieter Amazon und VMware eingegangen.

\subsection{Abgrenzung}
Dieses Dokument beschränkt sich auf das Angebot von Workstation in der \Gls{glos:cloudLabel}. Auf andere \Gls{glos:cloudLabel}-Angebote, wie das Hosting von Servern oder jegliche andere \Gls{glos:cloudLabel}-Dienste, wird nicht eingegangen.

Auf \Gls{byodLabel} wird ebenfalls nicht eingegangen. Die Diskussion beschränkt sich auf \gls{glos:onPremiseLabel} und \Gls{glos:cloudLabel}-Lösungen. Der Umfang von \Gls{byodLabel} wäre zu gross.

Es wird auch nicht auf rechtliche Aspekte eingegangen, da diese abhängig vom Standort und dem Tätigkeitsbereich der Firma sehr variieren können.

Auch findet sich in diesem Dokument keine konkrete Anleitung für die Bereitstellung und den Betrieb einer Workstation \gls{glos:onPremiseLabel} oder in der \Gls{glos:cloudLabel}.

% - nur Workstations (keine virtuelle Server etc.)
% - Bring your own Device wird ignoriert.
% - keine rechtliche Abklärungen
% - Keine konkrete Umsetzung
