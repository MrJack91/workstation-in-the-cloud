\chapter{Beschreibung der Aufgabe}

\section{Aufgabenstellung}
% aus Aufgabenstellung v2:
% (https://docs.google.com/document/d/1RX4SnkgstYHtR3L8P-43zTOZWYBdmZ10h4wnGvS7nMg/edit?usp=sharing)


\subsection{Ausgangslage}
Die meisten Menschen arbeiten gegenwärtig mit einer traditionellen „on-premise“ Computerlösung.
Ein Grossteil der Rechenleistung bleibt ungenutzt, da nur zu Peaks die volle Leistung benötigt wird.
Spart man Geld bei den Workstations, muss der Benutzer warten, sobald rechen-intensive Arbeiten durchgeführt werden. 

Mit einer Cloudlösung soll es möglich sein, dem User ständig so viel Rechenpower bereitzustellen, wie er benötigt.
Dies spart gesamt Ressourcen ein, da die Leistung zwischen den Benutzern geteilt werden kann.

\subsection{Ziele der Arbeit}
Die geschilderte Ausgangslage drängt den Gedanken an eine “soziale” Ressourcen Aufteilung auf.

\begin{itemize}
	\item Können, in Anbetracht von Kosten und Aufwand, verfügbare Ressourcen effizient, (aber gerecht) zur Verfügung gestellt werden?
	\item Welche Software für solche Ansprüche ist momentan auf dem Markt führend?
	\item Welche Hardware wird für die Bereitstellung der Cloud benötigt?
\end{itemize}


\subsection{Eingrenzung}
%todo Eingrenzung
- Abklärung Hosting von Working Station

\subsection{Abgrenzung}
%todo Abgrenzung
- nur Workstations (keine virtuelle Server etc.)  
- Bring your own Device wird ignoriert.  
- keine rechtliche Abklärungen  
- Keine konkrete Umsetzung  