\chapter{Schlussfolgerung}

% sicher folgende Fragen beantworten (aus der Aufgabenstellung):
% - Steht ein Umdenken von der bisherigen \gls{glos:onPremiseLabel} Strategie, zur Cloud-Philosophie an?
% - Die geschilderte Ausgangslage drängt den Gedanken, an eine “soziale” Ressourcen Aufteilung, auf. Können, in Anbetracht von Kosten und Aufwand, verfügbare Ressourcen effizient, (aber gerecht) zur Verfügung gestellt werden?

% Weitere einleitende Fragen aus der Aufgabenstellung:
% 	\item Was sind Vor- und Nachteile einer \Gls{glos:cloudLabel} Lösung.
% 	\item Welche Software für solche Ansprüche ist momentan auf dem Markt führend?
% 	\item Wie kann eine Workstation in der \Gls{glos:cloudLabel} angeboten werden?

% http://www.gfi.com/whitepapers/Hybrid_Technology.pdf
% Is a hybrid model onpremis/cloud a solution?

\section{Fazit}
In dieser Arbeit wurden \Gls{vdiLabel}-, \Gls{daasLabel}- und \gls{glos:onPremiseLabel}-Lösungen gegenübergestellt. Eine klare Aussage zu treffen, was eingesetzt werden soll, ist nicht möglich, da jede Umsetzung ihre Vor- und Nachteile besitzt.

Traditionell werden die Desktops auf einem \gls{glos:onPremiseLabel}-Rechner unter dem Arbeitstisch jedes Mitarbeiters betrieben. Diese verbinden sich auf die Server, welche im Rechenzentrum stehen.
Die Anschaffungskosten für Hardware sind überschaubar.
Auf dem Markt ist viel Know-How vorhanden, weshalb diese Lösung sich für \Glspl{kmuLabel} eignet.
Hingegen ist im Unterhalt viel Aufwand nötig. Updates sind aufwändig einzuspielen und die Administration der Rechner intensiv.

Um die Kontrolle zu vereinfachen, begannen Firmen rechenstarche Server anzuschaffen.
Die Mitarbeiter verbinden sich über einen \Gls{glos:thinClientLabel} auf die Server, welche die Arbeit verrichten. Mit dieser Cloud-Lösung können die Ressourcen besser genutzt und Updates zentral verwaltet werden. Die Anschaffungskosten sind jedoch sehr hoch und es braucht Spezialisten, welche die Wartung betreiben.
Ein wesentlicher Vorteil ist, dass man von zuhause und unterwegs auf seiner bekannten Umgebung arbeiten kann und die Daten die eigene Unternehmung nicht verlassen.
Geeignet ist diese Lösung für grosse Unternehmen. Einerseits wegen den hohen Anschaffungskosten, aber auch um die Daten besser zu schützen.

Denn grössten Nutzen hat man, wenn zusätzlich zu dieser Lösung, ein Outsourcing stattfindet und so die Kosten für den Unterhalt minimiert werden.

Als drittes wurde \Gls{daasLabel} vorgestellt. Einige Firmen begannen sich die Hardware für eine \Gls{vdiLabel}-Lösung anzuschaffen um die Cloud anderen Unternehmen anzubieten. Diese Unternehmen verbinden sich über einen \Gls{glos:thinClientLabel} auf die externe Infrastruktur und bezahlen einen monatlichen Betrag abhängig der Leistung die sie benötigen. Damit werden die horrenden Anschaffungskosten umgangen und die Flexibilität der Cloud bleibt beibehalten. Die Kosten sind somit fest im Budget kalkulierbar und es wird kein Expertenwissen benötigt. Damit eignet sich \Gls{daasLabel} für \Glspl{kmuLabel}, sowie für grosse Firmen. Für Unternehmen mit sensiblen Daten, stellt das externe Hosting ein Problem dar, da die Daten die eigene Firma verlassen.

\section{Hybrid-Lösung}
Möchte man die Flexibilität der Cloud, hat jedoch Daten welche die eigene Firma nicht verlassen dürfen, ist es auch möglich eine Hybrid-Lösung einzusetzen. Sprich es wird grundsätzlich auf \Gls{daasLabel} gesetzt, man betreibt parallel dazu jedoch noch eigene Server für die sensiblen Daten.
Dies birgt jedoch weitere sicherheitstechnische Probleme, da die Daten für den externen Zugriff freigegeben werden müssen.

%\section{neue Fragen}
% die sich aus der Arbeit ergeben haben


% print full glossary
\glsaddall
