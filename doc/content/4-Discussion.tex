
\chapter{Diskussion}
Im vorher gegangen Kapitel wurden die on-premis Lösung beschrieben, sowie das auslagern der Workstations in die Cloud auf fremden und eigenen Servern. In diesem Passus werden die verschiedenen Vorgehen gegeneinander verglichen. Das Kapitel soll helfen, sich für eine Lösung zu entscheiden.

% http://www.stratacore.com/blog/choosing-between-vdi-and-daas
% DaaS vs VDI
% + daas
	% Load balancing
	% Software and security updates
	% Installations
	% Network maintenance 
	% Schneller aufgesetzt
	% better cost benefit / ROI
% + vdi
	% slightly faster, because local datacenter, no firewall
	% recommended when data security is top priority

\section{On-Premis vs In-The-Cloud}
%sl
% sources:
% http://searchvirtualdesktop.techtarget.com/feature/DaaS-vs-VDI-comparison-highlights-benefits-of-cloud-desktops
% http://searchvirtualdesktop.techtarget.com/tip/Virtual-desktop-benefits-that-sell-VDI

% kosten abschätzbar
% günstiger für viele workstations
% günstiger da outsourcing möglich
% kosten abschätzbar
% thin clients günstiger

% pilots einfach
% incremental scaling
% Ein image für alle benutzer
% Updates einfacher. Eine Zentrale stelle für updates
	% Evt muss für updates neue hardware angeschaft werden. Mehr ram etc.
% Updates/Deployments sind lighning fast
% Fehler auffinden einfacher, da nur in einem system gesucht werden muss
% Backup an einer zentraler stelle

% Go green. Thin clients brauchen weniger strom
% Platform unabhängig. Egal ob windows, mac, linux
% Work from abroad
% Performance besser genutzt / Reductions in computing hardware needs. 
% Improved workplace/user flexibility,


% Greater control how you secure your desktop
% e.g prevent users from copy files to local machine

% daas
% einfacher
% günstiger

Als erster Schritt muss man eine On-Premis zu einer In-The-Cloud Lösung vergleichen.
In diesem Abschnitt werden die Vor- und Nachteile diskutiert. Unterteilt wird in die Bereiche Kosten, Aufwand, Unterhalt und Sicherheit.

\subsection{Kosten}
%http://betanews.com/2013/11/04/comparing-cloud-vs-on-premise-six-hidden-costs-people-always-forget-about/
% costs can heavily depend!
% to calculate costs of on premis soly on new hard-/software is like judging a book by its cover
% Same for cloud. They only calculate montly costs
% What needs to be regarded in the calculation:
	% - Electricity costs. Powering and cooling. Costs hard to meassure. Ask google/facebook. They need to watch PUE (Power Usage Effectivness)
		% More important the bigger you get.
	% - Bandwidth: Server inhouse: only limitation is the network infastructure. Ask a trusted consultant on what bandwidth you need for the cloud.
		% Check http://bandwidthpool.com/ for what you need. Login required.
	% - Cloudserver outbound bandwidth: You pay as you requst data
		% You can load as many data on the server, but pulling them out costs bandwidth
		% Understandable, you save money by not hosting them by yourself
		% not a big problem for virtual desktops
	% - Five year rule: On average, you need to replace 24/7 servers every 5 years
		% Cost diagram: http://www.softwareadvice.com/tco/ for SaaS, but maps to DaaS.
		% DaaS, higher recuring costs. but no 5 year hit-backs
	% - Downtime costs: The more they costs, the more it worth going into the cloud
		% Downtime cost calculator: http://www.visionsolutions.com/Solutions/Disaster-Recovery-toolkit-downtime-calc.aspx
		% Accross USA & EU, each year  $26.5 Billion USD is lost because of IT downtime

Im Kapitel \ref{fig:analyse} wurde aufgezeigt, dass es kostspielig sein kann, auf eine Cloud-Lösung umzustellen. Es müssen deshalb konkrete Bedürfnisse vorhanden sein, damit sich eine Umstellung rechtfertigt.

Bei der Berechnung der Kosten wird oft der Fehler gemacht, dass nur die Hard- und Software Kosten bei on-premis berücksichtigt werden, und die monatlichen Kosten bei der Cloud-Lösung.
Dass ist jedoch das selbe, wie wenn ein Buch nur nach dessen Deckblatt bewertet wird.
Es gibt diverse andere Kosten die es zu beachten gilt.

\subsubsection{Elektrizität}
Kosten welche oft nicht berücksichtig werden sind jene für die Elektrizität. Diese können unterteilt werden in die für den Stromverbrauch der Geräten, sowie in die Stromkosten welche durch die Kühlung anfallen. 

Unter den gesamten Ausgaben für die Elektrizität sind die Kosten für die Server oft schwer raus zu lesen, wenn keine Messwerkzeuge installiert sind. Der Vorteil von einem externen Anbieter besteht darin, dass diese Umtriebe bereits in den monatlichen Umtrieben enthalten sind. 

Je grösser die Firma wird, desto grösser wird dementsprechend auch der Stormverbrauch. Dies wird noch verstärkt oder abgeschwächt nach dem Arbeitsgebiet auf dem das Geschäft operiert. Bei einem service-orientiertem Angebot ist der Konsum deutlich höher. 

Um die effektivität des Stromverbrauches eines Datacenters zu messen gibt es das \Gls{pueLabel} Verhältnis. Dazu wird der gesamte Verbrauch einer Firma geteilt durch Stromkosten der IT gerechnet.
\[PUE = \frac{Total\: Stromkosten}{Stromkosten\: der\: IT}\]
Je näher der Wert bei 1 liegt, desto besser ist er. Microsoft gibt an, einen \Gls{pueLabel} Wert zwischen 1.13 bis 1.2 zu besitzen und Google arbeitet mit 1.14.

Den \Gls{pueLabel} Wert auf ein erträgliches Mass zu reduzieren kann zum Teil schwierig oder kostspielig sein, wenn man seine ganze Infrastruktur umstellen muss. Eine Cloud-Lösung stellt ein geeignetes Mittel dazu dar.

\subsubsection{Datenraten}
Ein weiterer Kostenpunkt sind die Datenraten und Netzwerkkosten. Bei einer Cloud-Lösung können beträchtliche Datenraten anfallen, wenn Datein, Programme und weiteres ausser Haus gespeichert werden. Wenn jedoch nur der Bildschirm übertragen werden halten sich die Kosten in Grenzen. 

Für das Betreiben der Server in-house fallen zwar keine Internet kosten an, jedoch wird das eigene Netzwerk stark belastet. 

Für eine neue Firma ist es günstiger in der Cloud zu arbeiten da keine grossen Netzwerkosten anfallen.
Hat man hingegen bereits ein bestehendes, gut ausgebautes Netzwerk kann dies eine Hürde darstellen beim Umstieg in die Cloud.

Zur Berechnung der benötigten Datenraten gibt es Berechnungssoftware oder auch online Services zu finden. Ein Beispiel ist die Seite \url{<http://bandwidthpool.com/>}.

Ein versteckter Kostenpunkt bei gewiesen Clouddiensten sind die auswärts Datenraten. Man kann gratis und so viele Daten man möchte auf einen Server laden, diese anschliessend ab zugreifen kostet.
Auch diese Kosten sind für DaaS Lösungen nicht sehr hoch, müssen jedoch trotzdem eingerechnet werden.

\subsection{5-Jahre-Regel}
Im Durchschnitt müssen Hardware-Komponenten alle fünf Jahre ausgetauscht werden.

\subsection{Ressourcen (Hardware), Thin Client}

\subsection{Data Protection}
% Wenn Daten nur in der Cloud, ist ein verlorenen Laptop (Daten mässig) nicht schlimm.

\subsection{Einschränkungen für Benutzer}
\subsection{Kein Offline-Access}
\subsection{Zusammenspiel Soft-/Hardware}

\section{VDI vs DaaS}
% technisches knowhow nötig
% knowhow teuer
% http://www.forbes.com/sites/benkepes/2013/11/06/death-to-vdi-or-daas-or-whatever-its-called-this-week/
	% does not actually solve many problems
	
% http://www.virtualqube.com/blog/cloud-based-vdi-vs-daas-is-there-a-difference/
% Lizenzprobleme mit Windows

%http://www.brianmadden.com/blogs/ brianmadden/archive/2014/01/27/building-vdi-for-100-000-users-why-huge-daas-providers-don-t-use-dell-and-hp-and-why-they-can-do-vdi-cheaper-than-you.aspx
% DaaS is cheaper than VDI
% They do it at scale. 100'000 Users. 20'000 CPU's. 400'000 GB RAM.
% What do you do. buy 2'000 Rack Servers from HP/Dell? They use alot of space. Stacked they are 300 feet high.
% When you open a server, what do you see? Each one has its own power supply. You dont need that.
% Buy a coustom motherboard. 
	% No VGA Port, no Display Controller
	% No need for SSD Boxes. Take chip out and solder them directly on the motherboard. Save money for the chips and the casing.
% You need a metal casing for each one.
% No GPU Cards. Order chips and build them in.
% Consumes less power. less heat. less space
% Google & Facebook do it like that
% DaaS Provider pay less for hardware und real-estate and electirizity to power and cool


\section{Bereitstellung: Arbeitsplatzes in der Cloud}



\section{Mobile Alternativen}
% mh

\subsection{VHD auf externem Medium}

\subsection{Chromebooks}
