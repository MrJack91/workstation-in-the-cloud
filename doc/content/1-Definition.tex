\chapter{Beschreibung der Aufgabe \& Definition}

\section{Aufgabenstellung}
% aus Aufgabenstellung v2:
% (https://docs.google.com/document/d/1RX4SnkgstYHtR3L8P-43zTOZWYBdmZ10h4wnGvS7nMg/edit?usp=sharing)


\subsection{Ausgangslage}
Die meisten Menschen arbeiten gegenwärtig mit einer traditionellen „on-premise“ Computerlösung.
Ein Grossteil der Rechenleistung bleibt ungenutzt, da nur zu Peaks die volle Leistung benötigt wird.
Spart man Geld bei den Workstations, muss der Benutzer oft warten, wenn rechen-intensive Arbeiten durchgeführt werden. 

Mit einer Cloudlösung soll es möglich sein, dem User ständig so viel Rechenpower bereitzustellen, wie er benötigt.
Dies spart Ressourcen ein, da die Leistung zwischen den Benutzern geteilt werden kann.

\subsection{Ziele der Arbeit}
Steht ein Umdenken von der bisherigen „on-premise“ Strategie, zur Cloud-Philosophie an?

Die geschilderte Ausgangslage drängt den Gedanken an eine “soziale” Ressourcen Aufteilung auf.

\begin{itemize}
	\item Können, in Anbetracht von Kosten und Aufwand, verfügbare Ressourcen effizient, (aber gerecht) zur Verfügung gestellt werden?
	\item Welche Software für solche Ansprüche ist momentan auf dem Markt führend?
	\item Welche Hardware wird für die Bereitstellung der Cloud benötigt?
\end{itemize}



\subsection{Eingrenzung}

\subsection{Abgrenzung}



\section{Definitionen}

\subsection{Cloud}
% allgemeine Definition von Cloud

Unter dem Begriff Cloud fallen versch. Definitionen. Folgend zwei Beispiele, die einen Eindruck der öffentlich bekannten Definition machen sollen.

\begin{quote}
	Unter Cloud Computing (deutsch etwa: Rechnen in der Wolke) versteht man das Speichern von Daten in einem entfernten Rechenzentrum (umgangssprachlich: „Ich lade das Bild mal in die Cloud hoch.“), aber auch die Ausführung von Programmen, die nicht auf dem lokalen Rechner installiert sind, sondern eben in der (metaphorischen) Wolke (englisch cloud). (Quelle: Wikipedia\footcite{Cloud_Computing__Wikipedia_2014-10-03})
\end{quote}


\begin{quote}
	Die Cloud ist eines der ältesten Sinnbilder der Informationstechnik und steht als solches für Rechnernetze, deren Inneres unbedeutend oder unbekannt ist. - (Quelle: The Wall Street Journal\footcite{The_Internet_Industry_Is_on_a_Cloud_--_Whatever_That_May_Mean_-_WSJ_2014-10-03})
\end{quote}






\subsubsection{Platform-as-a-Service (PaaS): Workstation}
% spezifisch für unser Fall: Workstation

\section{Vorliegendes Material}

\subsection{Angebotsübersicht (Preise für Cloud Lösungen)}
